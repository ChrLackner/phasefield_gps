% Created 2024-08-19 Mon 09:25
% Intended LaTeX compiler: pdflatex
\documentclass[11pt]{article}
\usepackage[utf8]{inputenc}
\usepackage[T1]{fontenc}
\usepackage{graphicx}
\usepackage{longtable}
\usepackage{wrapfig}
\usepackage{rotating}
\usepackage[normalem]{ulem}
\usepackage{amsmath}
\usepackage{amssymb}
\usepackage{capt-of}
\usepackage{hyperref}
\usepackage{my_defines}
\author{Christopher Lackner}
\date{\today}
\title{Notes on Grand Potential Formulation}
\hypersetup{
 pdfauthor={Christopher Lackner},
 pdftitle={Notes on Grand Potential Formulation},
 pdfkeywords={},
 pdfsubject={},
 pdfcreator={Emacs 30.0.50 (Org mode 9.6.7)}, 
 pdflang={English}}
\makeatletter
\newcommand{\citeprocitem}[2]{\hyper@linkstart{cite}{citeproc_bib_item_#1}#2\hyper@linkend}
\makeatother

\usepackage[notquote]{hanging}
\begin{document}

\maketitle
\tableofcontents

todo: Document strict mass conservation
todo: Anisotropy

\section{Grand Potential}
\label{sec:org9b950ef}

Most of this is taken from (\citeprocitem{1}{Aagesen et al. 2018}).

This phase-field model describes \(N\) possible phases and \(K\) chemical species.
The individual phases are represented y a set of nonconserved order parameters
\(\eta_{i}\), \(i=1,\ldots,N\).

We track the number density \(\rho\) of each solute species at each position.
Assuming each chemical species has the same atomic volume \(V_a\), \(K-1\) variables are required and the \(K\) species is considered the solvent.
The number density of species \(k\) is related to its local molar fraction \(c_k\) by

$$
\rho_k = \frac{c_k}{V_m}
$$

with \(V_m\) the molar volume.

The total grand potential \(\Omega\) of the system is defined as

$$
\Omega = \int_V (\omega_{mw} + \omega_{grad} + \omega_{chem}) \, dx
$$

\subsection{Multiwell potential}
\label{sec:orgf73f7a9}


(\citeprocitem{3}{Moelans, Blanpain, and Wollants 2008}) proposed a homogeneous free energy \(\omega_{mw}\) of the form 

$$
\omega_{mw} = m \left(\frac{1}{4} + \sum_{i=1}^N \left( \frac{\eta_i^4}{4} - \frac{\eta_i^2}{2} \right) + \sum_{i=1}^N \sum_{j=i+1}^N \gamma_{ij} \eta_i^2 \eta_j^2 \right)
$$

It has multiple degenerate minima located at

$$
(\eta_1, \ldots, \eta_N) = ( \pm 1, 0, \ldots, 0 ), ( 0, \pm 1, \ldots, 0), \ldots, ( 0, \ldots, 0, \pm 1)
$$

Interfacial anisotropy can be included by making \(\gamma_{ij}\) dependent on the interface orientation.

\subsection{The gradient contribution}
\label{sec:orgfe87ecd}

The gradient contribution \(\omega_{grad}\) is defined as

$$
\omega_{grad} = \frac{\kappa}{2} \sum_{i=1}^N \left| \nabla \eta_i \right|^2
$$

\subsection{The chemical contribution}
\label{sec:orgf6f89d5}

The chemical contribution \(\omega_{chem}\) is defined as

$$
\omega_{chem} = \sum_{i=1}^N h_i \omega_i
$$

with \(h_i\) an interpolation function for phase \(i\) and \(\omega_i\) the grand potential density of phase \(i\).

The interpolation function \(h_i\) is defined as (\citeprocitem{4}{Moelans 2011})

$$
h_i = \frac{\eta_i^2}{\sum_{j=1}^N \eta_j^2}
$$

The grand potential density \(\omega_i\) of phase \(i\) is defined as

$$
\omega_i = \frac{f_i}{V_m} - \sum_{k=1}^{K-1} \rho_k \mu_{k}
$$

where \(f_i\) is the Helmholtz free energy density of phase \(i\) and \(\mu_k\) is the chemical potential difference between species \(k\) and the solvent.

\subsection{Free energy functions}
\label{sec:orgc61a2eb}

\subsubsection{Ideal solution model}
\label{sec:org64a5d99}

\todo{Binary for now, generalize later. -> only one species with concentration $c$ and solvent $1-c$.}

In an ideal solution model, the free energy is a weighted average of the pure substance free energies of A and B plus an entropy of mixing term (\citeprocitem{5}{Plapp 2011}):

$$
f_i(T,c) = (1-c) f_i^A(T) + c f_i^B(T) + s_i RT \left[ c \ln c + (1-c) \ln (1-c) \right]
$$
with \(c\) the atomic fraction of \(B\). (\(f_i\) in \(J/\text{mol}\))

With \(c_k\) the concentration of the \$k\$-th species for \$k=1 \dots{} K-1 and

$$
c_K = 1 - \sum_{k=1}^{K-1} c_k
$$

\begin{align*}
f_i(T,c) =& \sum_{k=1}^{K} c_k f_i^k(T) + s_i RT \sum_{k=1}^{K} c_k \ln c_k
\end{align*}

with \(s_i\) is the site constant for phase \(i\) and \(R\) the gas constant.

With defining \(\epsilon_i^k = f_i^k(T) - f_i^K(T)\), the chemical potential is


$$
\mu_k = \frac{\partial f_i(T,c)}{\partial c_k} = \epsilon_i^k + s_i R T \ln{\left(\frac{c_k}{1-\sum_{l=1}^{K-1} c_l}\right)} = \epsilon_i^k + s_i R T \ln{\left(\frac{c_k}{c_K}\right)}
$$

with \(\mu\) in \(J/\text{mol}\).

this can be inverted to yield the concentration in each phase as a function of \(\mu\):

with \(\alpha_i^k := \frac{\mu_k - \epsilon_i^k}{R T}\)

$$
c_k^i(\mu_k) = \frac{\left(1-\sum_{l=1, l != k}^{K-1} c_l^i \right) e^{\alpha_i^k}}{1 + e^{\alpha_i^k}}
$$.

This can be used to define the interpolated composition and susceptibility:

$$
c_k(\eta, \mu_k) = \sum_{i=1}^N h_i(\eta) c_k^i(\eta, \mu_k)
$$

$$
\chi_{kl}(\eta, \mu) = \frac{\partial \rho_k}{\partial \mu_l} = \frac{1}{V_m R T} \sum_{i=1}^N h_i(\eta) c_i(\mu) (1-c_i(\mu))
$$

for details see \hyperref[sec:orga844052]{Derivation of susceptibility}.

\section{Evolution Equations}
\label{sec:org4961ef6}

\subsection{Phase order parameters}
\label{sec:org246fdcc}

Each order parameter \(\eta_i\) evolves by an Allen-Cahn equation

$$
\frac{\partial \eta_i}{\partial t} = -L \frac{\delta \Omega}{\delta \eta_i}
$$

with

$$
L = \frac{ \sum_{i=1}^N \sum_{j=1 i!=j}^N L_{ij} \eta_i^2 \eta_j^2 }{ \sum_{i=1}^N
\sum_{j=1 i!=j}^N \eta_i^2 \eta_j^2 }
$$

with \(L_{ij}\) a mobility coefficient for the interface between phases \(i\) and \(j\).

\begin{align*}
\frac{\delta \Omega}{\delta \eta_i} &= \frac{\partial \Omega}{\partial \eta_i} + \sum_{k=1}^K \frac{\partial \Omega}{\partial \mu_k}\frac{\partial \mu_k}{\partial \eta_i}
\end{align*}

\begin{align*}
\frac{\partial \Omega}{\partial \mu_k} = -\sum_{i=1}^N \frac{\partial \rho_k}{\partial \mu_k} h_i = -\sum_{i=1}^N \chi_{kk} h_i
\end{align*}

we currently approximate the derivative of the chemical potential with respect to the order parameter as the difference between the chemical potential at \(\eta_i = 1\) and \(\eta_j = 1\).

So this gives in total:

\begin{align*}
\frac{\partial \eta_i}{\partial t} = -L \left( \frac{\partial \Omega}{\partial \eta_i} - \sum_{k=1}^K \sum_{j=1}^N \chi_{kk} h_j \chi_{kk} \frac{\partial \mu_k}{\partial \eta_i} \right)
\end{align*}

\subsection{Chemical potentials}
\label{sec:org3bc8283}

In the case when all interdiffusivities are zero we get

$$
\chi_{kk} \frac{\partial \mu_k}{\partial t} = \nabla \cdot M_{kk} \nabla \mu_k - \sum_{i=1}^N \frac{\partial \rho_k}{\partial \eta_i} \frac{\partial \eta_i}{\partial t}
$$

with \(M_{kk} = D \chi_{kk}\), \(D\) the diffusivity and \(\chi_{kl}\) the susceptibility defined as
$$
\chi_{kl} = \frac{\partial \rho_k}{\partial \mu_l}
$$

for details see \hyperref[sec:orgb8d1e50]{Derivation of evolution equation for chemical potential}.

\section{Calculation of Model Parameters}
\label{sec:orgbd4346e}

from (\citeprocitem{3}{Moelans, Blanpain, and Wollants 2008}):

This only holds for \(\gamma \approx 1.5\)!
$$
\kappa = \sigma_{gb} l_{gb} \frac{\sqrt{f_{0,\text{interf}}(\gamma)}}{g(\gamma)} \approx
\frac{3}{4} \sigma_{gb} l_{gb}
$$

$$
L = \frac{mu_{gb}}{l_{gb}} \frac{g(\gamma)}{\sqrt{f_{0,\text{interf}}(\gamma)}} \approx \frac{4}{3} \frac{\mu_{gb}}{l_{gb}}
$$

$$
m = \frac{\sigma_{gb}}{l_{gb}} \frac{1}{\sqrt{f_{0,\text{interf}}(\gamma)}} \approx \frac{3}{4} \frac{1}{f_{0, \text{saddle}}(\gamma)} \frac{\sigma_{gb}}{l_{gb}} = 6 \frac{\sigma_{gb}}{l_{gb}}
$$

\section{Anisotropy}
\label{sec:org3e3dd88}

From (\citeprocitem{2}{McFadden et al. 1993}):

We define the diffusion coefficient \(\kappa\) to be dependent on the orientation of the interface: \(\kappa = \kappa(\theta)\) with \(\theta\) the angle of the interface:

$$
\theta_i = \arctan{\left(\frac{\partial_y \eta_i}{\partial_x \eta_i}\right)}
$$

\section{Appendix}
\label{sec:org9afb58f}

\subsection{Table of Units}
\label{sec:org242c115}

\begin{center}
\begin{tabular}{lll}
Quantity & Unit & SI\\[0pt]
\hline
\(\eta\) & - & -\\[0pt]
\(\rho\) & \(\text{mol}/\text{m}^3\) & \(10^{30} \text{mol}/\text{m}^3\)\\[0pt]
\(\mu\) & \(\text{J}/\text{mol}\) & \(10^3 \text{J}/\text{mol}\)\\[0pt]
\(f\) & \(\text{J}/\text{mol}\) & \(10^3 \text{J}/\text{mol}\)\\[0pt]
\(c\) & - & -\\[0pt]
\(T\) & K & K\\[0pt]
\(R\) & \(\text{J}/\text{mol K}\) & \(10^3 \text{J}/\text{mol K}\)\\[0pt]
\(s\) & - & -\\[0pt]
\(V_m\) & \(\text{m}^3/\text{mol}\) & \(10^{-6} \text{m}^3/\text{mol}\)\\[0pt]
\(D\) & \(\text{m}^2/\text{s}\) & \(10^{-12} \text{m}^2/\text{s}\)\\[0pt]
\(\kappa\) & \(\text{J}/\text{m}^2\) & \(10^{-6} \text{J}/\text{m}^2\)\\[0pt]
\(L\) & \(\text{J}/\text{m}^2\) & \(10^{-6} \text{J}/\text{m}^2\)\\[0pt]
\(m\) & \(\text{J}/\text{m}^3\) & \(10^{-6} \text{J}/\text{m}^3\)\\[0pt]
\end{tabular}
\end{center}


\subsection{Inversion of chemical potential}
\label{sec:orgc74b4cc}

\begin{align*}
\mu_k = \epsilon_i^k + s_i R T \ln{\left(\frac{c_k}{1-\sum_{l=1}^{K-1} c_l}\right)} = \epsilon_i^k + s_i R T \ln{\left(\frac{c_k}{c_K}\right)} \\
\frac{\mu_k - \epsilon_i^k}{s_i R T} = \ln{\left(\frac{c_k}{c_K}\right)} \\
\exp{\frac{\mu_k - \epsilon_i^k}{s_i R T}} = \frac{c_k}{c_K} \\
c_k = c_K \exp{\frac{\mu_k - \epsilon_i^k}{s_i R T}}
\end{align*}

using 
$$
c_K = 1 - \sum_{l=1}^{K-1} c_l
$$

and defining \(\alpha_i^k := e^{\frac{\mu_k - \epsilon_i^k}{s_i R T}}\) we get

\begin{align*}
\frac{1+\alpha_i^k}{\alpha_i^k} c_k = 1-\sum_{l=1, l!=k}^{K-1} c_l
\end{align*}


\subsection{Derivation of susceptibility}
\label{sec:orga844052}

\begin{align*}
\chi_{k,l}(\eta, \mu) = \frac{\partial \rho_k}{\partial \mu_l} = \frac{1}{V_m} \frac{\partial c_k(\eta, \mu)}{\partial \mu_l} = \frac{1}{V_m} \sum_{i=1}^N h_i(\eta) \frac{\partial c_k(\mu)}{\partial \mu_l} = \\
 = \frac{1}{V_m} \sum_{i=1}^N h_i(\eta) \frac{\partial}{\partial \mu_l}\left( \frac{\exp{\frac{\mu_k - \epsilon^k_i}{s_i R T}}}{1 + \exp{\frac{\mu_k - \epsilon^k_i}{s_i R T}}} \right)
\end{align*}

define
$$
\alpha^k_i := \frac{\mu_k - \epsilon^k_i}{s_i R T}
$$

then

\begin{align*}
\frac{\partial c_k}{\partial \mu_l} &= \delta_{kl} \frac{\partial c_k}{\partial \alpha_i^k} \frac{\partial \alpha_i^k}{\partial \mu_k} = \frac{\exp{\alpha^k_i} (1+\exp{\alpha^k_i}) - \exp{\alpha^k_i} \exp{\alpha^k_i}}{(1+\exp{\alpha^k_i})^2} \frac{1}{s_i R T} = \\
&= \frac{1}{s_i R T} \frac{\exp{\alpha^k_i}}{(1+\exp{\alpha^k_i})^2}
\end{align*}
and

$$
c_k (1-c_k) = \frac{\exp{\alpha^k_i}}{1+\exp{\alpha^k_i}} \frac{1+\exp{\alpha^k_i} - \exp{\alpha^k_i}}{1+\exp{\alpha^k_i}} = \frac{\exp{\alpha^k_i}}{(1+\exp{\alpha^k_i})^2}
$$

therefore

$$
\chi_{kl}(\eta, \mu) = \delta{kl} \sum_{i=1}^N h_i(\eta) \frac{1}{V_m s_i R T} c^i_k(\mu) (1-c^i_k(\mu))
$$




\subsection{Derivation of evolution equation for chemical potential}
\label{sec:orgb8d1e50}

From the generalized diffusion equation

$$
\frac{\partial \rho_k}{\partial t} = \nabla \cdot \sum_{k=1}^{K-1} M_k \nabla \mu_k
$$

where mobility coefficient \(M_k\) is of dimensions (energy x length x time\()^{-1}\)
From this we need to derive a evolution equation for the chemical potential \(\mu_k\).

Note that in typical evaolution equation
$$
\frac{\partial c}{\partial t} = \nabla \cdot M_k \nabla \mu_k
$$
the mobility coefficient \(M_k\) is of dimensions length x (energy x time)\(^{-1}\).

The time derivative of the density can be expressed as

$$
\frac{\partial \rho_k}{\partial t} = \sum_{i=1}^{K-1} \frac{\partial \rho_k}{\partial \mu_i} \frac{\partial \mu_i}{\partial t} + \sum_{i=1}^{N} \frac{\partial \rho_k}{\partial \eta_i} \frac{\partial \eta_i}{\partial t}
$$

If (as in our case)

$$
\frac{\partial \rho_k}{\partial \mu_i} = \delta_{ik}
$$

then

$$
\frac{\partial \rho_k}{\partial t} = \chi_{kk} \frac{\partial \mu_k}{\partial t} + \sum_{i=1}^{N} \frac{\partial \rho_k}{\partial \eta_i} \frac{\partial \eta_i}{\partial t}
$$

resulting in the evolution equation

$$
\chi_{kk} \frac{\partial \mu_k}{\partial t} = \nabla \cdot M_k \nabla \mu_k - \sum_{i=1}^N \frac{\partial \rho_k}{\partial \eta_i} \frac{\partial \eta_i}{\partial t}
$$



\section{Bibliography}
\label{sec:org5518afe}

\begin{hangparas}{1.5em}{1}
\hypertarget{citeproc_bib_item_1}{Aagesen, Larry K., Yipeng Gao, Daniel Schwen, and Karim Ahmed. 2018. “Grand-Potential-Based Phase-Field Model for Multiple Phases, Grains, and Chemical Components.” \textit{Phys. Rev. E} 98 (2): 023309. \url{https://doi.org/10.1103/PhysRevE.98.023309}.}

\hypertarget{citeproc_bib_item_2}{McFadden, G. B., A. A. Wheeler, R. J. Braun, S. R. Coriell, and R. F. Sekerka. 1993. “Phase-Field Models for Anisotropic Interfaces.” \textit{Phys. Rev. E} 48 (3): 2016–24. \url{https://doi.org/10.1103/PhysRevE.48.2016}.}

\hypertarget{citeproc_bib_item_3}{Moelans, N., B. Blanpain, and P. Wollants. 2008. “Quantitative Analysis of Grain Boundary Properties in a Generalized Phase Field Model for Grain Growth in Anisotropic Systems.” \textit{Phys. Rev. B} 78 (2): 024113. \url{https://doi.org/10.1103/PhysRevB.78.024113}.}

\hypertarget{citeproc_bib_item_4}{Moelans, Nele. 2011. “A Quantitative and Thermodynamically Consistent Phase-Field Interpolation Function for Multi-Phase Systems.” \textit{Acta Materialia} 59 (3): 1077–86. \url{https://doi.org/10.1016/j.actamat.2010.10.038}.}

\hypertarget{citeproc_bib_item_5}{Plapp, Mathis. 2011. “Unified Derivation of Phase-Field Models for Alloy Solidification from a Grand-Potential Functional.” \textit{Phys. Rev. E} 84 (3): 031601. \url{https://doi.org/10.1103/PhysRevE.84.031601}.}\bigskip
\end{hangparas}
\end{document}